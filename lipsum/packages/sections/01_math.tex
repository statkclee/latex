
수학 문서 작업을 위해서 많이 사용되는 수식관련 팩키지는 다음이 있다.

\begin{verbatim}
	\usepackage{amsmath}
	\usepackage{amssymb}
	\usepackage{amsfonts}
	\usepackage[T1]{fontenc}
	\usepackage[urw-garamond]{mathdesign}
	\usepackage{garamondx}
\end{verbatim}


\subsection{면적 공식}

\begin{equation} \label{eq1}
	\begin{split}
		A & = \frac{\pi r^2}{2} \\
		& = \frac{1}{2} \pi r^2
	\end{split}
\end{equation}


\subsection{오일러 공식}

\begin{equation} \label{eu_eqn}
	e^{\pi i} + 1 = 0
\end{equation}

\subsection{수학 기호}

\begin{itemize}
	
	\item  \verb| \mathbb{N} | : $\mathbb{N}$
	\item  \verb| \mathbb{R} | :  $\mathbb{R}$
	\item  \verb| \Rrightarrow | :  $\Rrightarrow$
	\item  \verb| \therefore | :  $\therefore$
	\item  \verb| \bigstar | :  $\bigstar$ 
	
\end{itemize}

